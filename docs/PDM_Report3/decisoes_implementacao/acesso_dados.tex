\subsection{\texttt{Acesso a dados}}
%%\label{sec:dataAccess}

Durante o desenvolver da aplica��o sempre existiu uma grande necessidade de armazenar os dados temporariamente para que estes pudessem ser reutilizados mais tarde e garantir assim uma obten��o mais r�pida e sem necessidade de gasto de recursos ou da exist�ncia deles, como � o caso de uma liga��o � Internet.
Outro dos grandes problemas existentes estava tamb�m relacionado com a submiss�o de mensagens tardia, ou seja, o agendamento das mesmas devido � inexist�ncia de conectividade.

Com a introdu��o dos \texttt{ContentProviders} foi ent�o poss�vel implementar mecanismos que respondessem �s necessidades encontradas.

Com a nova vers�o da aplica��o � adicionado ent�o um \texttt{ContentProvider} - \texttt{TwitterProvider} que permite armazenar dados numa base de dados \textit{SQLite}. Este \textit{provider} � gerido pelo \texttt{TwitterHelper} que � uma implementa��o de \texttt{SQLiteOpenHelper} que � respons�vel pelas ac��es de \textit{Create}, \textit{Drop} e \textit{Upgrade} do \textit{provider}.

Para poder utilizar o \textit{provider} s�o ent�o disponibilizadas tr�s classes de contracto de comunica��o com o provider, para uniformizar a sua utiliza��o. S�o estas o \texttt{TweetContract}, \texttt{TweetPostContract} e o \texttt{UserContract}.

Uma vez que um \texttt{ContentProvider} disponibiliza apenas um recurso sobre o qual podemos realizar opera��es CRUD, � ent�o criada uma camada de acesso a dados \textit{(DAL - Data Access Layer)} que permite efectuar as opera��es CRUD mas com uma camada de abstrac��o, facilitando assim as \textit{queries} de acesso a dados e a manipula��o dos dados retornados.