\subsection{\texttt{IntentService}}
\label{sec:intent}
Para implementar um \textit{Started Service} podemos seguir duas abordagens:
\begin{itemize}
	\item Implementar um servi�o que estenda de \texttt{Service} e que implemente a interface \texttt{Runnable};
	\item Implementar um servi�o que estenda de \texttt{IntentService}.
\end{itemize}

No trabalho apresentado � necess�rio criar dois servi�os do tipo \textit{Started Service}:
\begin{itemize}
	\item \texttt{StatusUpload} \ref{sec:statusUpload}
	\item \texttt{TimelinePull} \ref{sec:timelinePull}
\end{itemize}

Para implementar estes servi�os, do tipo \textit{Started Service} foi necess�rio estender de \texttt{IntentService}.

Um \texttt{IntentService} permite criar um \textit{Started Service} que j� � um \texttt{Runnable}, ou seja, o c�digo j� � executado assincronamente e o programador n�o tem que se preocupar com aspectos de implementa��o/execu��o de uma classe do tipo \texttt{Runnable}.


Como tal, ao implementar um servi�o que estenda de \texttt{IntentService} � necess�rio ter os seguintes aspectos em considera��o:
\begin{itemize}
	\item Cada \texttt{Intent} recebido em \texttt{onStartCommand()} � colocado numa fila;
	\item A \textit{worker thread} encaminha os \textit{intent}, um de cada vez, para \texttt{onHandleIntent()};
	\item O servi�o � terminado quando a fila fica vazia;
	\item O m�todo \texttt{onBind()} retorna \texttt{null}.
\end{itemize}