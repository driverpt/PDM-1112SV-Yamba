Na terceira etapa do trabalho pr�tico da disciplina de Programa��o em Dispositivos M�veis pretende-se optimizar a aplica��o desenvolvida nas duas etapas anteriores com recurso aos seguintes componentes/\textit{features} da plataforma \textit{Android}:
\begin{itemize}
	\item Armazenamento em mem�ria persistente com recurso � base de dados \textit{SQLite};
	\item Monitoriza��o de conectividade do equipamento (e tipo de conectividade, nomeadamente \textit{WiFi});
	\item Interac��o entre aplica��es para partilha de recursos;
	% A descomentar se o Luis implementar \Utiliza��o de \textit{Widgets};
	% A descomentar se o Luis implementar \Utiliza��o de \textit{Widgets};
\end{itemize}

Para proceder a estas optimiza��es s�o efectuadas as seguintes altera��es ao trabalho:

\begin{itemize}
	\item A lista de mensagens mostradas no \textit{timeline} � guardada numa base de dados \textit{SQLite}. Na vers�o anterior a lista era sempre constru�da a partir de objectos em mem�ria vol�til;
	\item Existe agora a possibilidade de fazer submiss�o de mensagens mesmo que n�o exista conectividade na aplica��o. Estas mensagens ficam tamb�m numa base de dados. Aquando da obten��o de conectividade s�o publicadas no servidor;
	\item Para partilhar uma mensagem atrav�s de \textit{email} o utilizador apenas precisa de efectuar um toque prolongado na lista de \textit{tweets}, aparecendo-lhe um menu contextual para o efeito, ou no detalhe de um \textit{tweet} com recurso ao menu da actividade.	
\end{itemize}

